\documentclass[a4paper]{article}
\usepackage{packages}
\title{Introdução à Teoria dos Conjuntos - Prova 2}
\author{Ariel Serranoni Soares da Silva  - Número USP: 7658024\\
Pedro Felizatto - Número USP:9794531\\
Pietro Mesquita Piccione - Número USP: 4630640\\
Mateus Schmidt Mattos Lopes Pereira - Número USP: 10262892\\
Luís Cardoso - Número USP: 4552403\\
Ariel Campêlo Viana Morais - Número USP: 9302177}
\date{\today}
\begin{document}
\maketitle
\section*{Observações iniciais}

As notas de aula a seguir foram produzidas com base na Seção 3 do Capítulo 12 de
\cite{jech}. A ideia do nosso trabalho é seguir fielmente o conteúdo abordado no
livro, inclusive mantendo a notação e numeração que são usados pelo autor.
Além disso, vamos incluir observações, soluções para os exercícios, e justificar
passos que ficaram em segundo plano no tratamento feito na obra mencionada.

\setcounter{section}{2}
\section{Árvores}

Assim como fizemos anteriormente com partições, vamos agora dedicar uma seção do
nosso trabalho para generalizar mais um objeto originário do
contexto de combinatória finita: Árvores.

\begin{definition}
  Uma \emph{árvore} é um conjunto ordenado \((T,\leq)\) tal que:
 \begin{enumerate}[(i)]
  \item \(T\) possui um menor elemento;
  \item para cada \(x\in T\), o conjunto \(\{y\in T\,\colon y<x\}\) é bem
    ordenado sob \(\leq\).
  \end{enumerate}
\end{definition}
Os elementos de \(T\) são chamados de \emph{nós}. Em particular,
o menor elemento de \(T\) é chamado  de \emph{raiz}.
Se \(x,y\in T\) são nós tais que  \(y<x\), dizemos que \(y\) é um
\emph{antecessor} de \(x\) e que \(x\) é um \emph{sucessor} de \(y\).

Seja \(x\in T\) um nó.  A \emph{altura} \(h(x)\) de \(x\) é o único ordinal
isomorfo ao conjunto bem ordenado \mbox{\(\{y\in T\,\colon
y<x\}\),} composto por todos os antecessores de \(x\). Observe que o Teorema 3.1
do Capítulo 6 garante que a altura está bem-definida como função em \(T\).
Além disso, se \(h(x)\) é um ordinal sucessor, dizemos que \(x\) é um
\emph{nó sucessor}. Caso contrário \(x\) é denominado \emph{nó limite}.
O \(\alpha\)-ésimo \emph{nível} de \(T\) é o conjunto
\(T_\alpha=\{x\in T \,\colon h(x)=\alpha\}\). A \emph{altura} \(h(T)\) da árvore
\(T\) é o menor ordinal \(\alpha\) tal que \(T_\alpha=\varnothing\).  

Um \emph{ramo} \(b\) de \(T\) é uma cadeia maximal em \(T\).  O \emph{comprimento}
\(\ell(b)\) de um ramo \(b\) é o tipo da ordem de \(b\). Note que para
todo ramo \(b\) de \(T\) temos \(\ell (b)\leq h(T)\).
Um ramo cujo comprimento é igual a \(h(T)\) é chamado de \emph{cofinal}.
Uma \emph{subárvore} \(T^\prime\) de \(T\) é um subconjunto
\(T^\prime\subseteq T\) tal que, para quaisquer \(x\in T^\prime\) e \(y\in T\),
temos que \(y<x\) implica \(y\in T^\prime\).
Sendo assim, \(T^\prime\) também é uma árvore quando ordenada por \(\leq\).
Ademais, para cada \(\alpha < h(T^\prime)\) temos que o \(\alpha\)-ésimo nível
de \(T^\prime\) é dado por \(T_\alpha^\prime= T_\alpha \cap T^\prime\).

 Para cada \(\alpha\leq h(T)\), o conjunto \(T^{(\alpha)}=\bigcup_{\beta <
   \alpha} T_\beta\) é uma subárvore de \(T\) com  \(h(T^{(\alpha)})=\alpha\).
 Se \(x\in T_\alpha\) então \(\{y\in T\,\colon y < x\}\) é um ramo de \(T^{(\alpha)}\) de
comprimento \(\alpha\); entretanto, se \(\alpha\) é um ordinal limite então
\(T^{(\alpha)}\) pode ter outros ramos de comprimento \(\alpha\).

Finalmente, um conjunto \(A\subseteq T\) é uma \emph{anticadeia} em \(T\) se
quaisquer elementos de \(A\) são incomparáveis. Isto é, se \(x,y\in A\) são tais
que \(x\not = y\), então \(x\not \leq y\) e \(y\not \leq x\).

A respeito dos conceitos introduzidos acima, é importante observar que todo
ramo ou subárvore de \(T\) contém a raiz. Também notamos que se \(x,y\in T\)
são nós tais que \(y< x\), então \(h(y) <h(x)\). Além disso, a prórpia definição
de \(h(T)\) nos dá que \(T_\alpha\not=\varnothing\) para todo ordinal
\(\alpha<h(T)\). A seguir, vamos resolver os
exercícios sugeridos pelo autor neste ponto do texto em \cite{jech}. Tais problemas
exploram algumas das propriedades que surgem imediatamente das
definições que apresentamos:

\begin{exercicio}
  Seja \((T,\leq)\) uma árvore. Mostre que:
  \begin{enumerate}[(i)]
  \item O único elemento \(r \in T\) tal que \(h(r)=0\) é a raiz. Em particular,
    \(T_0\not =\varnothing\);
  \item Se \(\alpha,\beta\) são ordinais tais que \(\alpha\not = \beta\), então \(T_\alpha\cap T_\beta =\varnothing\);
  \item \(T=T^{(h(T))}=\bigcup_{\alpha < h(T)}T_\alpha\);
  \item Um nó \(x\in T\) é um nó sucessor se, e somente se existe um único nó \(y\in T\) tal que
    \begin{equation}\label{propsuc}
      y < x \text{ e não existe } z \text{ tal que } y<z<x.
    \end{equation}
    Se \(x\) é um nó sucessor então o nó
    \(y\) que satisfaz \eqref{propsuc}  é chamado de \emph{antecessor imediato} de
    \(x\), e dizemos que \(x\) é \emph{sucessor imediato} de \(y\).
    Note que cada nó sucessor possui um único antecessor imediato,
    mas pode possuir múltiplos sucessores imediatos;
    \item Se \(x,y\in T\) são nós tais que \(y< x\), então existe um único \(z\in T\) tal que \(y< z\leq x\) e
      \(z\) é um sucessor imediato de \(y\);
    \item \(h(T)=\sup\{\alpha +1\,\colon
      T_\alpha\not=\varnothing\}=\sup\{h(x)+1\,\colon x\in T\}.\)
    \end{enumerate}
  \end{exercicio}
  
\begin{proof}[Solução]\hfill
  \begin{enumerate}[(i)]

  \item Seja \(r\in T\) tal que \(h(r)=0\), então temos que \(\{x\in T\,\colon
    x<r\}=\varnothing\), da onde segue que \(r\) é o menor elemento de \(T\), a raiz.
    Agora suponhamos \(r_1\, r_2\in T\), de modo que \(h(r_1)=h(r_2)=0\). Então \(r_1\) e
    \(r_2\) são raízes de \(T\). Neste caso,  para todo \(x\in T\) temos que
    \(r_1\leq x\). Em particular temos que \(r_1\leq r_2\).
    Analogamente, obtemos que \(r_2\leq r_1\). Como \(\leq\) é
    uma ordem, segue que \(r_1=r_2\). Para \(r\in T\) raiz, temos \(h(r)=0\), logo
    \(r\in T_0\). Logo, \(T_0\not=\varnothing\).
    
  % Por definição, \(T\) possui um menor
  % elemento, o que implica que \(T\) possui uma raiz \(r\). Para verificar a
  % unicidade de \(r\), suponha que \(T\) possui duas raízes \(r_1\) e \(r_2\)
  % distintas. Neste caso,  para todo \(x\in T\) temos que
  %  \(r_1\leq x\). Em particular temos que
  %  \(r_1\leq r_2\). Analogamente, obtemos que \(r_2\leq r_1\). Como \(\leq\) é
  %  uma ordem, segue que \(r_1=r_2\), contradição. Finalmente notamos que,
  %  como \(r\) é o menor elemento de \(T\), segue que \(\{x\in T\,\colon
  %  x<r\}=\varnothing\). Como este último fato implica que \(h(r)=0\), concluímos que
  %  \(r\in T_0\). Logo, \(T_0\not=\varnothing\).

  \item Suponha que existe \(x\in T_\alpha\cap T_\beta\). Neste caso, temos que
    \(\alpha=h(x)=\beta\), o que contradiz a unicidade de \(h(x)\).

  \item Para a primeira igualdade, note que \(T^{(h(T))}\subseteq T\) uma vez
    que \(T^{(h(T))}\) é definido como uma união de subconjuntos de \(T\). Por
    outro lado, seja \(x\in T\) e suponha que \(x\not\in T^{(h(T))}\). Daí,
    segue da definição de \(T^{(h(T))}\) que não existe \(\alpha < h(T)\) tal
    que \(h(x)=\alpha\). Isso implica
    que \(h(x)\geq h(T)\). Absurdo. Como a segunda igualdade segue diretamente da
    definição de \(T^{(h(T))}\), a prova está completa. 

  \item Suponha que \(x\) é um nó sucessor. Neste caso, temos por definição
    que \(h(x)=S(\alpha)\) para algum ordinal \(\alpha\). Vamos mostrar que
    \(T_\alpha\) é não vazio. Assumindo o contrário, teremos que
    \(h(x)>\alpha\geq h(T)\), que é um absurdo. Além disso,
    se  existe \(z\) tal que \(y<z<x\), então
    \[h(y)=\alpha<h(z)<S(\alpha)=h(x),\] absurdo. Finalmente, suponha que
    existem \(y_1\)
    e \(y_2\) distintos satisfazendo \eqref{propsuc}. Neste caso, o conjunto \(S=\{w\in
    T\,\colon w <x\}\) não é bem ordenado pois o subconjunto
    \(\{y_1,y_2\}\not = \varnothing\) de \(S\) não possui um menor elemento.

    Para verificar a implicação reversa, observamos que se existe \(y\) satisfazendo
    \eqref{propsuc}, então \(h(x)=S(h(y))\). Do contrário, existe \(z\in T\) tal
    que \(h(y)<h(z)<h(x)\), implicando que \(y<z<x\). Absurdo.

  \item Em primeiro lugar, suponha por absurdo que não existe tal elemento
    \(z\). Seja \(\alpha=h(x)\) e considere a subárvore \(T^{(\alpha)}\) de \(T\).
    Seja \(b=\{w \in T \,\colon w<x\}\) e recorde que \(b\) é um ramo de
    \(T^{(\alpha)}\). Note que \(y\in b\) e que \(b\) tem comprimento \(\alpha\). Se
    não existe \(z\), então \(b\) não é maximal. \textcolor{red}{Ariel S: Quase
      certeza que esse item não tá certo! Alguém sabe arrumar?}

  \item  Por definição, \(h(T)\) é o menor ordinal
    \(\alpha\) tal que \(T_\alpha=\varnothing\). Sendo assim, podemos ver que
    \[h(T)\geq \alpha +1 \text{ para cada } \alpha \text{ tal que }
      T_\alpha\not=\varnothing,\]
    e que
    \[h(T)\geq h(x)+1\text{ para cada } x\in T.\]
    Daí segue que
    \[h(T)\geq \sup\{\alpha+1\,\colon T_\alpha\not=\varnothing\}\text{ e que }
      h(T)\geq\sup\{h(x)+1\,\colon x\in T\}.\]
    Por fim, suponha que
    \[h(T)> \sup\{\alpha+1\,\colon T_\alpha\not=\varnothing\}\text{ ou } h(T)
      >\sup\{h(x)+1\,\colon x\in T\}\]
    e note que em ambos os casos, segue que \(T_\beta=\varnothing\) para algum \(\beta<
    h(T)\). Absurdo.\qedhere
\end{enumerate}
\end{proof}
\begin{exercicio}
    Seja \((T,\leq)\) uma árvore. Mostre que:
    \begin{enumerate}[(i)]
    \item Cada cadeia em \(T\) é bem-ordenada;
    \item Se \(b\) é um ramo de \(T\) e \(x\in b\), e \(y<x\), então \(y\in b\);
    \item Se \(b\) é um ramo em \(T\), então \(\card{b\cap T_\alpha}=1\) para
      \(\alpha <\ell (b)\) e \(\card{b\cap T_\alpha}=0\) para \(\alpha
      >\ell (b)\). Conclua que \(\ell (b)\leq h(T)\);
    \item \(h(T)=\sup\{\ell (b)\,\colon b \text{ é um ramo de } T\}\);
    \item \(T_\alpha\) é uma anticadeia para cada \(\alpha < h(T)\).
     \end{enumerate}
  \end{exercicio}
  \begin{proof}[Solução]\hfill
    \begin{enumerate}[(i)]
      \item  Seja \(C\subseteq T\) uma cadeia e seja
        \(\varnothing\not = S\subseteq C\). Vamos provar que \(S\) possui um menor
        elemento.

        Seja \(y\in S\). Se \(y\) é o menor elemento de \(S\) então a prova está
        completa. Caso contrário, temos que existe \(x\in S\) tal que \(x<y\). Assim,
         o conjunto \(S_{y} =\{z\in S\,\colon z<y\}\) é não vazio. Além disso, como
         \(S_y\) está contido em \(\{z\in T\,\colon z<y\}\), que é bem ordenado por
         definição, obtemos que \(S_y\) possui um menor elemento \(m\).

         Por fim, vamos mostrar que \(m\) é o menor elemento de \(S\). De fato, se
         \(s\in S\) então ou \(y \leq s\) ou \(s<y\), pois \(S\subseteq C\), que é uma cadeia.
         Se \(y \leq s\) então \(m<y\leq s\). Por outro lado se
         \(s<y\) então \(s\in S_{y}\), e assim \(m\leq s\). De toda forma, temos que
         \(m\leq s\). Logo,  \(m\) é o menor elemento de \(S\).

       \item Suponha que \(y\not\in b\). Vamos mostrar que \(y\) é comparável
        a todos os elementos de \(b\). Se \(z\in b\) e \(z < x\), temos que
        \(y\) e \(z\) são comparáveis pois ambos pertencem ao conjunto \(\{w\in
        T\,\colon w< x \}\), que é bem ordenado, e em particular, linearmente
        ordenado. Caso \(x<z\), obtemos que \(y<z\) por transitividade. Assim,
        concluímos que \(b\cup\{y\}\) é uma cadeia e portanto \(b\) não é
        maximal. Absurdo.

        \item \textcolor{red}{Ariel S: Aqui o item (ii) estava duplicado e está
            faltando o (iii) !!!}
        
      \item Como observamos anteriormente, temos que  \(\ell (b)\leq h(T)\) para
        todo ramo \(b\) de \(T\). Este fato nos permite concluir que
        \[h(T)\geq\sup\{\ell(b)\,\colon b \text{ é um ramo de } T\}\]
        Assim nos resta verificar que \(h(T)\) é de fato o menor dos majorantes
         de \(\{\ell(b)\,\colon b \text{ é um ramo de } T\}\). Isto é, vamos
         provar que se \(\alpha\geq  \ell (b)\)
        para todo ramo \(b\) de \(T\), então \(\alpha\geq h(T)\).

        De fato, como \(\alpha\geq \ell (b)\) para todo ramo \(b\) de \(T\),
         segue do item anterior que \(b\cap T_\alpha=\varnothing\)  para
         todo \(b\)  ramo de \(T\).

         Como para todo nó \(x\in T\) existe um ramo \(b_x\) tal que
         \(x\in b_x\) e \(T_\alpha\cap b= \varnothing\) para todo ramo
         \(b\subseteq T\), concluímos que \(T_\alpha=\varnothing\).
         Este último fato implica que \(h(T)\leq \alpha\), pois \(h(T)\) é o
         menor ordinal \(\beta\) que satisfaz \(T_\beta =\varnothing\).  

      \item Seja \(\alpha < h(T)\) e assuma que existem \(x,y\in T_\alpha\)
          distintos tal que \(x\) e \(y\) são comparáveis.
          Suponha sem perda de generalidade que
          \(x<y\). Neste caso, temos que \(x<y\) mas \(h(x)=h(y)\). Absurdo, pois
          \(x<y\) implica que \(h(x) < h(y)\).\qedhere

    \end{enumerate}
  \end{proof}

  Agora vamos olhar para alguns exemplos de árvores:
  \begin{exemplo}\hfill
    \begin{enumerate}[(a)]
    \item Todo conjunto bem-ordenado \((W,\leq)\) é uma árvore. Sendo assim,
      podemos pensar em árvores como generalizações de boas ordens. A altura
      \(h(W)\) é o tipo de ordem de \(W\), e o único ramo de \(W\) é o próprio
      \(W\), que é cofinal.
    \item Seja \(\lambda\) um número ordinal e seja \(A\) um conjunto não-vazio.
      Defina \(A^{< \lambda}=\bigcup_{\alpha < \lambda} A^\alpha\) como o
      conjunto de todas as sequências transfinitas de elementos de \(A\) com
      comprimento menor que \(\lambda\). Considere \(T=A^{< \lambda}\) e o
      ordene segundo \(\subseteq\). Assim, para quaisquer \(f,g\in T\),
      temos \(f\leq g\) se, e somente se \(f\subseteq g\), o que significa que
      \(f=\rest{g}{\dom(f)}\). É fácil verificar que \((T,\subseteq)\) é uma árvore:
      Primeiro, note que a sequência vazia é o menor elemento de \(T\). Além
      disso, para cada \(f\in T\), o conjunto \(\{x\in T\,\colon x < f\}\) é
      bem-ordenado pois, para qualquer um de seus subconjuntos, a sequência de
      menor comprimento é o menor elemento.

      Também é imediato que para cada \(f\in T\),
      temos que \(h(f)=\alpha\) se, e somente se \( f\in A^\alpha\). Isto é,
      \(T_\alpha =A^\alpha\). Ademais, se \(\alpha\) e \(\beta\) são ordinais
      tais que \(\alpha =\beta +1\) e \(f\in A^\alpha\), a sequência
      \(\rest{f}{\beta}\) é o antecessor imediato de
      \(f\) e os elementos de \(f\cup \{\langle \beta ,\alpha \rangle\}\) são
      sucessores imediatos de \(f\).

      Por fim, observemos que existe uma  correspondência biunívoca entre os
      ramos de \(T\) e as funções de \(\lambda\) em \(A\):
      se \(f\in A^\lambda\) então \(\{{\rest{f}{\alpha}\,\colon\alpha <\lambda}\}\) é
      um ramo em \(T\). Por outro lado, se \(b\) é um ramo em
      \(T\) então \(b\) é um sistema compatível de funções
      e \(f=\bigcup_{g\in b} g\in A^\lambda\). Vale notar que todos os ramos de \(T\) são
      cofinais, uma vez que possuem comprimento \(\lambda=h(T)\).
      
    \item Generalizando o exemplo anterior, se
      \(T\subseteq A^{<\lambda}\) é uma subárvore de 
      \((A^{<\lambda},\subseteq)\) com altura \(h(T)=\alpha\), então existe uma
      correspondência biunívoca entre os ramos de \(T\) e as funções
      \(f\in A^{<\lambda}\cup A^\lambda\) satisfazendo cada uma das seguintes
      propriedades:
      \begin{enumerate}[(i)]
      \item \(\rest{f}{\alpha}\in T\) para todo \(\alpha \in \dom(f)\);
      \item \(f\not\in T\) ou \(f\in T\) e \(f\) não possui sucessores em \(T\). 
      \end{enumerate}
     Nessa situação, costuma-se identificar um  ramo utilizando sua função correspondente.

   \item Consideremos o conjunto
    \(T\subseteq \Naturals^{<\omega}\)
     das sequências de números naturais que são finitas e
    decrescentes, ou seja, \(f\in T \) se, e somente se  \(f(i)>f(j)\)  para todo
    \(i<j<\dom(f)\in\Naturals\).  Então \((T,\subseteq)\) é uma subárvore de
    \((\Naturals^{<\omega},\subseteq)\). Note que a altura de  \(T\) é 
    \(\omega\). Além disso, temos pelo Exercício 2.8 do Capítulo 3 que não
    existe uma sequência de números naturais que seja infinita e decrescente.
    Assim conseguimos concluir que todos os ramos de \(T\) são finitos, o
    que implica que \(T\) não possui ramos cofinais.

  \item Seja \((R,\leq)\) um conjunto linearmente ordenado.
    Uma \emph{representação} de uma árvore \((T,\preceq)\)
    \emph{por intervalos} em \((R,\leq)\) é uma função \(\Phi\) tal que para
    quaisquer \(x,y\in T\),
    suas imagens \(\Phi(x)\) e \(\Phi(y)\) são intervalos em \((R,\leq)\) satisfazendo:
    \begin{enumerate}[(i)]
    \item \(x\preceq y\) se, e somente se \(\Phi(x)\supseteq\Phi(y)\); 
    \item \(x\) e \(y\) são incomparáveis se, e somente se \(\Phi(x)\cap\Phi(y) =\varnothing\).
    \end{enumerate}
    Em particular, as propriedades acima implicam que \((\Phi[T],\supseteq)\) é
    uma árvore isomorfa a \((T,\preceq)\). Por exemplo, se considerarmos
    a árvore \(S=\text{Seq}(\{0,1\})=\{0,1\}^{<\omega}\) ordenado por \(\subseteq\),
    então o sistema \(\langle D_s \,\colon s\in S\rangle\) construído no Exemplo
    3.18 do  Capítulo 10 é uma representação de \((S,\subseteq)\) por intervalos
    fechados na reta real.
    \end{enumerate}
  \end{exemplo}
  
  O estudo de árvores finitas é um dos  principais conceitos que aparecem em
  combinatória. Entretanto, não vamos nos aprofundar muito no tema. Ao invés
  disso, vamos investigar algumas propriedades árvores infinitas, nos preocupando
  especialmente com condições suficientes para que uma árvore possua um ramo cofinal.

  Para árvores cuja altura é um ordinal sucessor a resposta é óbvia: se \(h(T)=\alpha +1\),
  então \(T_\alpha\neq\varnothing\) e para qualquer \(x\in T_\alpha\) temos que
  \({{y\in T \,\colon y\leq x}}\) é ramo cofinal de \(T\). Assim, focaremos em
  árvores cuja altura é um limite. Por outro lado, o item (d) do Exemplo 3.2 nos diz que
  existem árvores de altura \(\omega\) que possuem apenas ramos finitos. O
  próximo teorema mostra que,  pensando na largura de uma árvore como um
  majorante para a cardinalidade de seus níveis, se uma aŕvore \(T\) de altura
  enumerável é suficientemente "esguia", então \(T\) possui um ramo cofinal.

  \begin{teo}[Lema de König]
  Seja \((T,\leq)\) uma árvore de altura \(\omega\). Se todos os níveis de \(T\) finitos, então
  \(T\) admite um ramo de altura \(\omega\).
  \end{teo}
\begin{proof}
  Em primeiro lugar, observamos que se  \(T\) é uma árvore, então cada nó de \(T\) possui
  um número finito de sucessores imediatos se, e somente se todos os níveis de
  \(T\) são finitos. Ancorados neste fato, vamos mostrar que se \(T\) é uma
  árvore de altura \(\omega\) tal que cada nó tem um número finito de sucessores
  imediatos,  então \(T\) tem um ramo infinito. De fato, vamos utilizar recursão
  para  construir uma sequência \(\langle c_n\rangle_{n=0}^{\omega}\)  de nós de \(T\) tal que,
  para todo \(n<\omega\), o conjunto \(\{a\in T\,\colon c_n\leq a\}\) é
  infinito.

  Seja \(c_0\) a raiz de T e  note que \(\{a \in T\,\colon c_{0} \leq a \}= T\)
  é infinito. Além disso, se \(\{a\in T\,\colon c_n\leq a\}\) é infinito para
  \(n<\omega\) e consideramos  o
  conjunto \(S_n\) dos sucessores imediatos de \(c_n\), então
  \[
  \{a\in T\,\colon c_n\leq a\} = \{c_n\} \cup\bigcup_{b\in S_n} \{a\in T\,\colon b\leq a\}.
  \]
 Daí segue que existe \(b\in S_n\) tal que o conjunto \(\{a\in T\,\colon
 b\leq a\}\) é infinito, assim podemos utilizar o Axioma da Escolha para
 tomar \(c_{n+1}\) como tal \(b\). 
 Finalmente, observamos que para cada \(\alpha<\omega\), o conjunto
 \[\{a\in T\,\colon  a\leq c_n \text{ para algum }
   n<\alpha\}=\bigcup_{n<\alpha}\{a\in T\,\colon a\leq c_n\}\] é
  um ramo de \(T^{(\alpha)}\) com altura \(\alpha\). Daí,
  segue por indução que \[\{a\in T\,\colon a\leq c_n \text{ para algum }
    n\in\Naturals\}=\bigcup_{n <\omega}\{a\in T\,\colon a\leq c_n\}\]
  é um ramo de \(T^{(\omega)}=T\).
  \end{proof}

    Apesar da prova acima conter um pouco mais de detalhe em relação àquela
    apresentada em \cite{jech}, não fizemos a prova em mais detalhe ainda para
    não tirar o foco da sua ideia principal, que é construir a sequência
    \(\langle c_n\rangle_{n=0}^\omega\) 'de baixo para cima', garantindo que
    sempre temos mais 'infinitos elementos' para serem adicionados e em seguida
    utilizar a sequência resultante para construir um ramo 'de cima para baixo'.

    Em especial, observe que usamos o Teorema da Recursão sem
    explicitamente especificar uma função \(g\) para
    calcular \(c_{n+1}\) a partir de \(c_n\). De fato, para consturir tal função precisamos
    aplicar o Axioma da Escolha: se \(k\) é uma função escolha para
    \(\mathcal{P}(T)\) e \(S_c\) é o conjunto de todos os sucessores imediatos
    de \(c\), então definindo \(g(c,n) \doteq k\big ( \{b\in S_c: \{a\in T: b\leq a\} \text{ é
      infinita}\}\big )\) temos que \(c_{n+1} = g(c_n, n)\). Nesse cenário, estamos
    conforme o Teorema da Recursão que vimos. No próximo exercício veremos que,
    se \(T\) é uma árvore com 'largura finita', então \(T\) possui um ramo
    cofinal não importando sua altura. Nesse sentido, o resultado a seguir é uma
    generalização do Lema de König.
  
  \begin{exercicio}
  Seja \(\kappa\) um cardinal infinito e seja \((T,\leq)\) uma árvore de altura
  $\kappa$. Prove que se todos os níveis de \(T\) são finitos, então \(T\)
  possui um ramo de tamanho $\kappa$.
\end{exercicio}
\textcolor{red}{Ariel S.: Não manjo de topologia então não tenho condições de
  verificar a prova a seguir propriamente. De toda forma, me parece que a ideia
  seria repetir a prova do Lema de König usando recursão transfinita.
  Alguém manja?}
\textcolor{green}{Ariel S.:OLHAR LINK: https://math.stackexchange.com/q/464694/253958 }

\begin{proof}[Solução]
  
  Para demonstrar tal resultado, iremos visualizar os ramos dessa árvore como um compacto contido num espaço topológico produto.
  De fato, temos que se \(X=\Pi_{\alpha<\kappa}(T_\alpha\cup\{0\})\), munido do produto das topologias discretas, então o conjunto \(C\) dos ramos da árvore é identificado como
  \(C=\{(x_\alpha)_{\alpha<\kappa}\in X : x_\beta<x_{\beta+1}\text{ pra todo }\beta\}\), onde definimos \(0>x\text{ pra todo }x\in T\).
  
  Não é difícil ver que \(C\) é fechado, portanto como \(X\) é compacto pelo Teorema de Tychonoff, \(C\) é compacto.
  Agora queremos construir \(F_\lambda\subset C\) fechados encaixantes, então teremos que \(\bigcap F_\lambda\neq\varnothing\), e assim \(x\in\bigcap F_\lambda\) será o ramo que desejamos.
  
  Defina \(F_0\doteq C\). Indutivamente suponhamos \(F_\alpha\) bem definido para \(\alpha<\lambda\), de maneira tal que \(\sup\{\ell(b):b\in F_\alpha\}=\kappa\) (Como visto no exercício 3.2(iv)).
  %falta definir ell(b)
  Como todo nó possui somente finitos sucessores imediatos, temos que \(A_\lambda=(\bigcap_{\alpha<\lambda} F_\alpha)\cap T_\lambda\) é finito. Agora existe \(a_\lambda\in A_\lambda\) de modo que quando definimos \[F_\lambda\doteq\{(x_\beta)\in C : x_\lambda=a_\lambda\}\bigcap_{\alpha<\lambda} F_\alpha\] temos que \(\sup\{\ell(b): b\in F_\lambda\}=\kappa\), e não é difícil ver que \(F_\lambda\) é fechado.
  Então temos que existe \(x\in\bigcap F_\lambda\), que será claramente o ramo de comprimento \(\kappa\).
  
  
  %comentário do ariel que precisamos revisar
  %\item O livro sugere a seguinte dica: Seja  \(U_{\alpha} = \{ x \in T_{\alpha} \vert  \mid \{y \in T \vert x \leq y \}\mid = \kappa\}\). Prove que \(\{\mid U_{\alpha} \mid \mid \alpha \leq \kappa \} \subseteq N \). Mostre que T tem exatamente m ramos de tamanho $\kappa$.
  %Contudo essa dica não necessariamente é verdade, pois o $\mid U_{\alpha} \mid$ é limitado. Considere uma árvore de altura $\omega$ binária, cada ponto de um nível gera dois ramos. Mas a dica falha para maiores alturas, por exemplo, pode-se criar uma arvore de altura etaomega que começa com um ramo, se divide em dois ramos no nivel eta1, então se divide em dois ramos no nivel eta2, e assim por diante. É claro que $\mid U_{\alpha}}$ é ilimitado, mesmo que todos os niveis sejam finitos. Um contra-exemplo similar também funciona quando a altura $\kappa$ tem cofinalidade %\omega%.
  \end{proof}
  

  Os dois últimos resultados nos mostram que para uma árvore \((T,\leq)\) cuja altura é um
  ordinal limite maior ou igual a \(\omega\) com níveis finitos, então \(T\)
  possui um ramo cofinal. A partir deste ponto, podemos pensar em generalizações
  desta condição. Por exemplo, é natural que nos perguntemos se $T$ é uma árvore
  de altura $\kappa$, onde cada nível tem cardinalidade estritamente menor de
  que \(\kappa\), é verdade que $T$ tem um
  ramo de comprimento $\kappa$?  A resposta para tal pergunta é \textbf{NÃO!}

  \begin{definition}
    Seja \(\kappa\) um cardinal não enumerável.
    Uma árvore de altura $\kappa$ é uma \textit{árvore de Aronszajn} se
    todos os os seus níveis tem cardinalidade estritamente menor do que \(\kappa\)
    e se não possui ramos de comprimento $\kappa$.
  \end{definition}

 A seguir, vamos estudar mais a fundo o caso em que \(\kappa=\omega_1\). Nesta
  situação, conseguimos construir árvores de Aronszajn, que servem como
  contra-exemplo  para a indagação que propusemos acima.
  
\begin{teo}
  Existem árvores de Aronszajn de altura $\omega_1$.
\end{teo}

\begin{proof}
  Vamos utilizar recursão transfinita para contruir, para cada ordinal \(\alpha
  <\omega_1\), o \(\alpha\)-ésimo nível de uma árvore de Aronszajn tal que:
  \begin{enumerate}[(i)]
      \item \(T_{\alpha} \subseteq \omega^{\alpha}\) e \(\card{T_{\alpha}} \le \aleph_0\);
      \item Se \(f \in T_{\alpha}\), então \(f\) é biunívoca e \(\omega - \ran f\) é infinito;
      \item Se \(f \in T_{\alpha}\) e \(\beta < \alpha\) então \(\rest{f} {\beta} \in T_{\beta}\);
      \item Para qualquer \(\beta < \alpha\), qualquer \(g \in T_{\beta}\), e
        qualquer \(X \subseteq \omega - \ran g\) finito, existe uma \(f \in
        T_{\alpha}\) tal que \(g \subseteq f\) de \(\ran f \cap X = \varnothing\).
      \end{enumerate}

     Em primeiro lugar, tomamos \(T_0=\{\varnothing\}\). Daí, assumindo que
     construímos o nível \(T_\alpha\) satisfazendo (i)-(iv). Então,
     definimos \[T_{\alpha+1}=\{g\cup\{\langle \alpha, a\rangle\}\,\colon g\in
       T_\alpha \text{ e } a\in \omega -\ran g\}\]. Vamos verificar que
     \(T_{\alpha+1}\) satisfaz (i)-(iv):
     \begin{enumerate}[(i)]
       \item \textcolor{red}{VERIFICAR!}
      \end{enumerate}
      
   Agora, vamos contruir \(T_{\alpha}\) no caso em que \(\alpha\) é um ordinal
   limite. Considere um ordinal \(\beta< \alpha\), um nó \(g \in T_{\beta}\) e
   um conjunto \(X \subseteq \omega - \ran g\) finito. Vamos construir uma
   função \(f = f(g, X)\) recursivamente da seguinte maneira: fixe uma sequência
   crescente \(\langle \alpha_n \rangle_{n=0}^{\infty}\) tal que \(\alpha_0  =\beta\)
   e \(sup\{\alpha_n\,\colon n \in \mathbb{N}\} = \alpha\). Seja
   \(f_0 = g \in T_{\alpha_0}\) e \(X_0 = X \subseteq \omega - ranf_0\). Tendo
   definido \(f_n \in T_{\alpha_n}\) e \(X_n = X \subseteq \omega - \ran f_n\)
   finito, nós primeiro tomamos um conjunto \(X_{n+1} \supset X_n\) finito tal
   que \(X_{n+1} = X \subseteq \omega - \ran f_n\). Note que este último passo é
   possível porque o conjunto \(\omega -\ran f_n\) é infinito por (ii).
   Feito isso, escolhemos \(f_{n+1} \in T_{\alpha_{n+1}}\) tal que \(f_{n+1}
   \supseteq f_n\) e  \(X_{n+1} \cap ranf_{n+1} = \varnothing\), o que possível
   por (iv). Seja \(f = \bigcup_{n = 0}^{\infty}f_n\). Claramente \(f\colon
   \alpha\to \omega\) e  \(f\) é biunívoca pois cada uma  das funções \(f_n\)
   também é. Além disso, temos que \(\ran f \cap (\bigcup_{n = 0}^{\infty} X_n)
   = \emptyset\) e, portanto, \(\omega - \ran f\) é infinito e \(\ran f \cap X =
   \varnothing\). Assim concluímos que \(f\) satisfaz (ii). Ademais, se  \(\beta <
   \alpha\), temos que  \(\rest{f} {\beta}= \rest{(f_{n})}{\beta}\) sempre que
   \(\beta < \alpha_n\), e portanto (iii) também é satisfeito.

   \textcolor{red}{Sobre o parágrafo de baixo: OI?}
    Colocamos esta $f = f(g, X)$ em $T_{\alpha}$ para cada $g \in \bigcup_{\beta < \alpha} T_{\beta}$ e cada $X \subseteq \omega - rang$. Portanto $(4)$ também é satisfeito. Como $|\bigcup_{\beta<\alpha}T_{\beta}| \le \sum_{\beta<\alpha}|T_{\beta}| \le \aleph_0$ (pela suposição indutiva $(1)$) e o número de subconjuntos finitos de $\omega$ é contável, o conjunto $T_{\alpha}$ é no máximo contável, e $(1)$ também é satisfeito.

    Finalmente, vamos mostrar que \(T = \bigcup_{\alpha < \omega_1}T_{\alpha}\)
    é uma árvore de Aronszajn.
    \textcolor{red}{Sobre o parágrafo de baixo: OI?}
      Claramente \(T\) é uma árvore por (iii) , cada nível é no máximo contável por \((i)\), e sua altura é \(\omega_1\) (por $(4)$, cada $T_{\alpha} \neq \emptyset$). Se $B$ fosse um ramo de comprimento $\omega_1$ em $T$, então $F = \bigcup B$ seria uma função um-pra-um de $\omega_1$ para $\omega$ (por $(2)$), uma contradição.

\end{proof}

 A questão da existência de árvores de Aronszajn é muito complicada e ainda não
 foi totalmente resolvida. De modo geral, dizemos que cardinais \(\kappa\)
 não-contáveis para os quais vale um  análogo do Lema de Kőnig, i.e., não
 existem árvores  de Aronszajn de altura $\kappa$; dizemos que tais cardinais
 possuem a  \textit{propriedade de árvore}. O exercício a seguir nos mostra que
 cardinais singulares não satisfazem tal requisito.

 \begin{exercicio}
  Construa uma árvore de Aronszajn de altura \(\aleph_{\omega}\). Generalize
  para um cardinal singular \(\kappa\) qualquer.
\end{exercicio}
\begin{proof}[Solução]
\textcolor{green}{Ariel S. : VER LINK:https://math.stackexchange.com/a/1686329/253958}
\end{proof}

 Por fim, notamos que que cardinais
 fortemente inacessíveis com a propriedade da árvore são precisamente os
 cardinais fracamente compactos definidos na seção anterior.
 \textcolor{red}{Seria bom elaborar um pouquinho nisso aqui também, acontece
   que não tenho a menor condição. Alguém leu a Seção 2?}

  \bibliographystyle{plain}
  \bibliography{sample}
\end{document}
